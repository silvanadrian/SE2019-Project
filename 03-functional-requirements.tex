\newpage
\section{Functional requirements}

The system contains a login function for both caseworkers and applicants. Caseworkers login to the system through an employee portal, while applicants login to the system using NemId.

\vspace{2mm}

Once the applicant is logged in, they are presented with a form through which they submit relevant information. The form contains text input fields and fields through which additional documentation can be attached to the application as pdf files.

\vspace{2mm}

The system assigns submitted applications to available caseworkers at random. Processed applications are archived and saved in an immutable state. Once logged in, the caseworker is presented with their assigned cases, as well as a way to search all archived cases. Caseworkers have access to all d

\vspace{2mm}

\vspace{2mm}

The OCM system checks for existing applications from applicant. If none are found the system annotate the case as new, otherwise it annotate the case as duplicate. The system sends application to a random caseworker.

\vspace{2mm}

If the caseworker takes an unprocessed case, the person will be presented with all the provided documents from the applicant. If the caseworker decides the documents are falsified, he changes the status to falsified and either gets unassigned from the case or takes further action by contacting the legal authorities and provide all the necessary information, that is requested. After the falsification is handled, the case change status to prosecuted.

After the verification of the authenticity of the documents, the caseworker decides if the doctor confirmation is confirming that the legal requirements of the child's health status is met and the necessary conditions are present. 
If the conditions are not met, the application gets rejected. 
If the documents provided does not fulfill the necessary conditions, the caseworker can send an email directly from the system to the applicant, requesting the necessary documents for further processing of the application.

When satisfactory is met and all the conditions and legal documents are met, the caseworker goes on to verify that the inputted numbers are correct.

If the caseworker picks an approval case, that is not handled by the caseworker himself. The caseworker goes through all documents provided and verifies that the applicant is permitted for loss of earnings. The caseworker goes on to verify that the inputted numbers are correct.
If the caseworker decides, that the applicant is not permitted for loss of earnings, the caseworker sets the case to rejected.

If the caseworker decides, that the applicant is permitted for loss of earning, the caseworker sets the case to approved.

---

The applicant is presented with a form, that contains all the required input fields and required fields for document upload. After submitting the application the applicant will receive an email from the system confirming the submission.

If the application needs some information, the applicant would receive a request per email and a link to the formula with the information that is previously provided. The application now has the possibility to input the requested information. 

If the application is rejected, the applicant receives cause of rejection per email.

If the application is approved, the applicant receives the acceptance confirmation per email. 

---

If the application is rejected or accepted twice, the system would send the decision to the applicant and archive the case. 

---

The supervisor can make an overview of all open cases and all archived cases. The supervisor can inspect the current information that belongs to a case and what caseworker it is assigned to.
\newpage
\begin{figure}[htb!]
	\includegraphics[width=\textwidth]{img/dcr}
	\caption{A very rough dcr graph}
\end{figure}